% Julian Rice
% Linguistics 165C: Semantics II Final
% uses dvi-ps-pdf-chain to compile

\documentclass[english, 11pt]{article}

%\usepackage{etex}
\usepackage[T1]{fontenc}
\usepackage{charter}
\usepackage[utf8]{inputenc}
\usepackage[usenames,dvipsnames,svgnames,table]{xcolor}
\usepackage{babel}
\usepackage[bottom = 1in, left=1in, right=1in]{geometry}
\usepackage{fancyhdr}	%for headers and footers
\pagestyle{fancy}
\usepackage{natbib}  %use if there are citations!
\usepackage{tipa}   %for IPA
\usepackage{amstext} 
\usepackage{lingmacros}
\usepackage{amsmath}
\usepackage{qtree}    %for trees
\usepackage{tikz}
\usepackage{caption}
\usepackage{tikz-qtree}
\usepackage{stmaryrd} 
\usepackage{arydshln} 
\usepackage{stmaryrd} %New for double brackets
\usepackage[colorlinks=false]{hyperref} %set colorlinks to false if you don't want colored links. comment it out if you don't want links at all.
%\usepackage{OTtablx}	%for OT tableaux. See below for example.
\usepackage{rotating}    %for angled text
\usepackage{pifont} %for symbols
\usepackage{dsfont}	%for more symbols
\usepackage{gb4e} 	%for linguistic examples and glosses

\newcommand{\dwork}{\text{\ding{83}}}
\newcommand{\xmark}{\text{\ding{53}}}
\newcommand{\onemeaning}{\text{\ding{172}}}
\newcommand{\twomeaning}{\text{\ding{173}}}

\newcommand{\vs}{\vspace{12pt}}  % Vertical skip of 12 pts
\newcommand{\underscore}{\underline{\hspace{0.5cm}}}  %0.5 cm long underline
\def\checkmark{\tikz\fill[scale=0.4](0,.35) -- (.25,0) -- (1,.7) -- (.25,.15) -- cycle;}

\begin{document}
\rfoot{\thepage}  %Page Number
\cfoot{}
\rhead{Julian M. Rice}	  %Right Header
\lhead{Scope Ambiguities in Japanese}  %left header
\title{Scope Ambiguities in Japanese\\
  \large Linguistics 165C: Semantics II
}	%Title
\author{Julian M. Rice}
\date{June 7th, 2019}
\maketitle
%\subsection{And you can have subsections too!}

\section{Introduction}
Scope ambiguities are a commonly seen and reviewed problem when covering the semantics of languages like English. The question then arises - what scope ambiguities exist for an entirely different language, like Japanese? This paper is aiming to introduce scope ambiguities and analyze different quantifiers in Japanese and provide details on the differences between these quantifiers. First and foremost, a quantifier is defined as a linguistic expression that specify or quantify a set (source). In English, examples of quantifiers include all, every, some, few, and exactly two. Quantifiers indicate the scope of a term, usually a noun or pronoun, to which it is attached to; this intrinsically brings potential sentence ambiguity to the table. Ambiguity can be considered the presence of two or more meanings in a sentence - two meanings represents two different logical forms (LF), or syntax trees that can be drawn from a single sentence. This is demonstrated in the following English sentence:

\begin {exe}
	\ex 
		\begin {xlist}
			\ex Every student met some professor.
			\begin {xlist} 
				\ex \textbf{LF1}: Every single student met some professor, not necessarily the same one.
				\ex \textbf{LF2}: Every single student met the same professor.
			\end {xlist}
			\ex Some student met every professor.
				\begin {xlist} 
					\ex \textbf{LF1}: Some specific student met every single professor.
					\ex \textbf{LF2}: For every professor there is, each one met some student.
				\end {xlist}
	\end {xlist}
\end {exe}
\begin{equation}
	\llbracket  \text{[Every\:student]}_{1}\:\text{[some\:professor]}_{2}\:[e_{1}\:\text{met}\:e_{2}] \rrbracket \quad
\end{equation}
\begin{equation}
	\llbracket  \text{[Some\:professor]}_{2}\:\text{[every\:student]}_{1}\:[e_{1}\:\text{met}\:e_{2}] \rrbracket \quad
\end{equation}
(1a-i) and (1b-i) show us that the first logical form, or LF1, places the emphasis on the subject. Equation (1) describes the underlying form for LF1 of (1a-i), whereas equation (2) describes the underlying form for LF2 of (1a-ii). The full syntax trees for (1a) is displayed on Figure 1 (next page), showing the appropriate quantifier raising for the differing logical forms. In LF1, \emph{some professor} is raised first, which is then followed by the raising of \emph{every student}. This creates a subject-wide interpretation, where 

% FIGURE 1: Scope Ambiguities in English (LF1 and LF2)
\begin{figure}[!h]
\centering
	\captionsetup{labelfont=bf}
	\caption[labelfont=bf]{Logical Form 1 and Logical Form 2 for (1a)}
	\begin{tikzpicture}[scale = 0.7]
		\Tree [.t  [.<e,t>,t \edge[roof];\node(2){every\:student_{1}}; ] [.t [.<e,t>,t \edge[roof];\node(4){some\:professor_{2}}; ] [.t \node(1){e_{1}}; [. <e,t> met \node(3){e_{2}}; ] ] ] ]]
		\draw [thin, dashed, ->] (1) .. controls +(south:3) and +(south:4.5) .. (2);
		\draw [thin, dashed, ->] (3) .. controls +(south:2) and +(south:3.5) .. (4);
	\end{tikzpicture} \hspace{50pt}
	\begin{tikzpicture}[scale = 0.7]
		\Tree [.t  [.<e,t>,t \edge[roof];\node(2){some\:professor_{2}}; ] [.t [.<e,t>,t \edge[roof];\node(4){every\:student_{1}}; ] [.t \node(1){e_{1}}; [. <e,t> met \node(3){e_{2}}; ] ] ] ]]
		\draw [thin, dashed, ->] (1) .. controls +(south:2) and +(south:2) .. (4);
		\draw [thin, dashed, ->] (3) .. controls +(south:2) and +(south:4.5) .. (2);
	\end{tikzpicture}
\end{figure}

%------------------------------------------------------------
\section{Scrambling and Ambiguity in Japanese}
%------------------------------------------------------------
\subsection{Standard SOV and Scrambled OSV Scope Ambiguity}
Japanese's subject-object-verb (SOV) sentence structure is canonical, meaning that only a subject wide interpretation (deemed as S>O) is available. The noun within the subject of the sentence acts as the entity that actively affects the object with the verb of the sentence. On the other hand, scrambling in Japanese, which involves switching the word order of a sentence without switching its core meaning, produces ambiguous sentences. This ambiguity only takes place when there is an existentially quantified subject and universally quantified object seen in the SOV sentence and scrambled - this is known as a QP-QP sentence (\cite{s1}).

\begin{exe}
	\ex 
	\begin{xlist}
		\ex[]{
		\gll Dareka-ga dono ringo-mo tabe-ta \\
			someone-Nom every apple eat-Past. \\
			\trans `Someone ate every apple.' }\label{1a}
		\ex[]{
		\gll Dono ringo-mo dareka-ga tabe-ta \\
			every apple someone-Nom eat-Past. \\
			\trans `Someone ate every apple (scrambled).' }\label{1b}
	\end{xlist}
\end{exe}
\begin {exe}
	\ex 
		\begin {xlist}
			\ex 
				\begin {xlist}
					\ex \textbf{S>O}: Person x ate every apple.\label{2ai}
					\ex \textbf{*O>S}: For every apple y, some person ate y.\label{2aii}
				\end {xlist}
			\ex
				\begin {xlist} 
					\ex \textbf{S>O}: Person x ate every apple.\label{2bi}
					\ex \textbf{O>S}: For every apple y, some person ate y.\label{2bii}
				\end {xlist}
		\end {xlist}
\end {exe}
In (2a) and (2b), we compare the standard SOV Japanese sentence (2a) with the scrambled OSV Japanese sentence (2b). Notice that the gloss produces the same sentence in English; as mentioned earlier, the core meaning of the sentence does not change regardless of scrambling. However, the key difference between (2a) and (2b) is demonstrated in (3). (3a) shows that the subject wide interpretation (S>O) for the SOV sentence is valid, however the object wide interpretation (O>S) is invalid, or unnatural. This implies that standardly written SOV Japanese sentences are not ambiguous. The aforementioned ambiguity with scrambled OSV sentences is demonstrated in (2b) and (3b). The universal quantifier in the object position has now been moved to the front of the sentence, and the existential quantifier has been moved to before the verb. (3b) shows that both S>O (LF1) and O>S (LF2) readings are valid, leading to scope ambiguity. 

\begin{exe}
	\ex 
	\begin{xlist}
		\ex[]{
		\gll Gakusei-no dareka-ga dono kudamono-mo tabe-ta \\
			student-Gen some-Nom every fruit eat-Past. \\
			\trans `Some student ate every fruit.' }\label{1a}
		\ex[]{
		\gll Dono kudamono-mo gakusei-no dareka-ga tabe-ta \\
			every fruit student-Gen someone-Nom eat-Past. \\
			\trans `Some student ate every fruit (scrambled).' }\label{1b}
	\end{xlist}
\end{exe}
\begin {exe}
	\ex 
		\begin {xlist}
			\ex 
				\begin {xlist}
					\ex \textbf{S>O}: Student x ate every fruit.\label{2ai}
					\ex \textbf{*O>S}: For every fruit y, some student ate y.\label{2aii}
				\end {xlist}
			\ex
				\begin {xlist} 
					\ex \textbf{S>O}: Student x ate every fruit.\label{2bi}
					\ex \textbf{O>S}: For every fruit y, some student ate y.\label{2bii}
				\end {xlist}
		\end {xlist}
\end {exe}
Sentences in (4) further demonstrate this addition of ambiguity through scrambling. This time, however, the noun \emph{student} is added to the subject, and object has also been modified to show the ambiguity in another context. (4a) is the SOV sentence and (5a) shows that, like seen in (2a) and (3a), subject wide interpretation is accepted but object wide interpretation is not. In the scrambled OSV sentence (4b), (5b) shows scope ambiguity by indicating that both interpretations are valid. The syntactic tree structure with quantify raising for (4b) is shown below in Figure 2 and its two respective underlying forms are shown in equation (3) and (4). Our current findings are summarized in Figure 3.
\begin{equation}
	\llbracket  \text{[Gakusei-no\:dareka-ga]}_{1}\:\text{[dono\:kudamono-mo]}_{2}\:[e_{1}\:e_{2}\:\text{tabe-ta}] \rrbracket \quad
\end{equation}
\begin{equation}
	\llbracket  \text{[Dono\:kudamono-mo]}_{2}\:\text{[gakusei-no\:dareka-ga]}_{1}\:[e_{1}\:e_{2}\:\text{tabe-ta}] \rrbracket \quad
\end{equation}

\begin{figure}[!h]
\centering
	\captionsetup{labelfont=bf}
	\caption[labelfont=bf]{Logical Form 1 and Logical Form 2 for (4b)}
	\begin{tikzpicture}[scale = 0.7]
		\Tree [.t  [.<e,t>,t \edge[roof];\node(2){Gakusei-no\:dareka-ga_{1}}; ] [.t [.<e,t>,t \edge[roof];\node(4){dono\:kudamono-mo_{2}}; ] [.t \node(1){e_{1}}; [. <e,t> \node(3){e_{2}}; {tabe-ta} ] ] ] ]]
		\draw [thin, dashed, ->] (1) .. controls +(south:3) and +(south:4.5) .. (2);
		\draw [thin, dashed, ->] (3) .. controls +(south:2) and +(south:3.5) .. (4);
	\end{tikzpicture} \hspace{20pt}
	\begin{tikzpicture}[scale = 0.7]
		\Tree [.t  [.<e,t>,t \edge[roof];\node(2){Dono\:kudamono-mo_{2}}; ] [.t [.<e,t>,t \edge[roof];\node(4){gakusei-no\:dareka-ga_{1}}; ] [.t \node(1){e_{1}}; [. <e,t> \node(3){e_{2}}; {tabe-ta} ] ] ] ]]		\draw [thin, dashed, ->] (1) .. controls +(south:1.5) and +(south:1.5) .. (4);
		\draw [thin, dashed, ->] (3) .. controls +(south:2) and +(south:4.5) .. (2);
	\end{tikzpicture}
\end{figure}
Figure 2 contains the tree structures for both interpretations of the scrambled OSV sentence in (4b). Both readings are seen in equations (3) and (4) and their explanations are listed in (5b). Sentence (4a) only has one logical form, meaning that although (5a-i) may be represented as a tree structure, (5a-ii) cannot. Notice that despite the scrambling, the underlying form displayed in Figure 2 for logical form 1 has the same underlying form as (4a). This means that scrambling \emph{maintains the original meaning} of the sentence \emph{and} it adds a second, new underlying form to the sentence.
\begin{figure}[h]
	\begin{center} \renewcommand*\arraystretch{1.2}
	\scalebox{1}[1]{\begin{tabular}[t]{|rrl||c|c|c|} \hline 
	\multicolumn{3}{|c||}{Subject - Object Structure} & \sc{SOV} & \sc{OSV}  \\[0.5ex]
  	 	\hline & Universal - Existential 	& & $\onemeaning$ & $\twomeaning$ \\
		\hline & Existential - Universal 	& & $\onemeaning$ & $\onemeaning$ \\
   	 	\hline 
	\end{tabular}} \renewcommand*\arraystretch{1} \end{center}
	\vspace*{-5mm}
	\captionsetup{labelfont=bf}
	\caption[labelfont=bf]{Japanese Scope Ambiguity Table Version I}
\end{figure}
\newline \newline
To clarify the in progress table above, we claim that, \emph{so far}, all universal quantifiers in Japanese that appear as a subject in a scrambled sentence cause scope ambiguity. \text{\onemeaning} represents sentences with a single underlying form, and \text{\twomeaning} represents sentences with two underlying forms (scope ambiguity). However, all other quantifier placements are non ambiguous, regardless of whether the sentence has been scrambled or not. Let us explore distributive and collective interpretations in both English and Japanese to see if this is really the case.

%------------------------------------------------------------
\subsection{Distributive and Collective Interpretations}
\subsubsection{English Interpretations}
% Define Distributive and Collective interpretations
% Write an example using 'all' vs 'every'
% Explain the characteristics of 'all' and 'every' in English with an example
In English, scope ambiguity often occurs as a result of a QP-QP sentence, or sentence that contains two quantifiers (one existential and one universal quantifier). However, when the object quantifier is \emph{all}, the object-wide interpretation becomes unnatural and the sentence loses its scope ambiguity. There are two types of interpretations that apply to universal quantifiers, and quantifiers can either support one or both of the interpretations when used in their respective language.
\begin {exe}
	\ex 
		\begin {xlist}
			\ex Some police officer arrested every criminal.
			\begin {xlist} 
				\ex \textbf{LF1}: A single, individual police officer arrested every criminal.
				\ex \textbf{LF2}: Every criminal was arrested by an officer, but not necessarily the same one.
			\end {xlist}
			\ex Some police officer arrested all criminals.
				\begin {xlist} 
					\ex \textbf{LF1}: A single, individual police officer arrested every criminal.
					\ex \textbf{LF2}: *All criminals were arrested by an officer, but not necessarily the same one.
				\end {xlist}
	\end {xlist}
\end {exe}
As seen in (6b), when the quantifier in the object position is \emph{all}, then the second logical form that places an emphasis on object-wide interpretation (O>S) becomes unnatural, making the subject-wide interpretation (S>O) the only reading for the sentence. On the other hand, the sentence in (6a) allows two logical forms that work, and its object quantifier is \emph{every}. 

The reason why \emph{every} in the object position allows scope ambiguity and \emph{all} does not allow scope ambiguity in the object position is because \emph{all} allows a \emph{collective interpretation}, whereas \emph{every} does not. A collective interpretation is when a word or collection of things can be referred to as a whole. The quantifier \emph{all} is an example of this, because when one says \emph{all}, it can refer to the entire group of whatever the quantifier is referring to as a whole. The other English quantifier, \emph{every}, is an example of a universal quantifier that does not allow a collective interpretation because saying 'every + N' (noun) cannot be referred to as a whole in any case. This is made clear in (6a). 

\begin {exe}
	\ex 
		\begin {xlist}
			\ex All the men carried a table upstairs.
			\begin {xlist} 
				\ex \textbf{Collective Interpretation (CI)}: The men carried one table together upstairs.
				\ex \textbf{Distributive Interpretation (DI)}: Each of the men carried one table upstairs.
			\end {xlist}
			\ex Every man carried a table upstairs.
				\begin {xlist} 
					\ex \textbf{Collective Interpretation (CI)}: *The men carried one table together upstairs.
					\ex \textbf{Distributive Interpretation (DI)}: Each of the men carried one table upstairs.
				\end {xlist}
	\end {xlist}
\end {exe}
The other kind of interpretation that applies to universal quantifiers is \emph{distributive interpretation}. This is defined as: a QP $\alpha$ occurring in a sentence \emph{S} supports a distributive interpretation if, under the reading, we can construe individual elements in the domain of $\alpha$ to co-vary with (the witness of) another quantifier $\beta$ that also occurs in the logical form of \emph{S} (\cite{s3}). The two variations in (7) show that \emph{all} in the subject position allows for collective interpretation, whereas \emph{every} in the subject position does not.
\begin{figure}[h]
	\begin{center} \renewcommand*\arraystretch{1.2}
	\scalebox{1}[1]{\begin{tabular}[t]{|rrl||c|c|c|} \hline 
	\multicolumn{3}{|c||}{English} & \sc{Distributive} & \sc{Collective}  \\[0.5ex]
  	 	\hline & All 		& & $\checkmark$ & $\checkmark$ \\
		\hline & Every 	& & $\checkmark$ & $\dwork$ \\
   	 	\hline 
	\end{tabular}} \renewcommand*\arraystretch{1} \end{center}
	\vspace*{-5mm}
	\captionsetup{labelfont=bf}
	\caption[labelfont=bf]{Universal Quantifiers and Interpretations (English)}
\end{figure}
\newline
It may be fair to assume that the fact that \emph{every} cannot be used for collective interpretations acts as a key reason as to why \emph{every} creates ambiguity when placed in the object position of an SOV sentence. The important points to take away from the data in (6) and (7) for QP-QP sentences is the following:
\begin{itemize}
	\item When \emph{all} is in the subject position, there is scope ambiguity.
	\item When \emph{all} is in the object position, there is no scope ambiguity.
	\item When \emph{every} is in the subject position, there is scope ambiguity.
	\item When \emph{every} is in the object position, there is scope ambiguity.
\end{itemize}

\subsubsection{Japanese Interpretations}
Given that we have a background of the nature behind universal quantifiers with different interpretations in English, it is time to observe Japanese universal quantifiers and find what traits they may that may relate or differentiate Japanese from English. For now, we will compare \emph{dono-N-mo} (every) with \emph{subete-no} (all) with sentences that each only contain a single quantifier.
\begin{exe}
	\ex 
	\begin{xlist}
		\ex[]{
		\gll Subete-no gakusei-wa piza-o tabe-owat-ta \\
			all-Gen student-Top pizza-Acc eat-finish-Past. \\
			\trans `All students ate a/the pizza.' }\label{1b}
		\ex[]{
		\gll Dono gakusei-mo piza-o tabe-owat-ta \\
			every student-Gen pizza-Acc eat-finish-Past. \\
			\trans `Every student ate a/the pizza.' }\label{1a}
	\end{xlist}
\end{exe}
\begin {exe}
	\ex 
		\begin {xlist}
			\ex Subete-no gakusei-wa piza-o tabe-owat-ta.
				\begin {xlist} 
					\ex \textbf{Collective Interpretation (CI)}: All the students ate a single pizza together.
					\ex \textbf{Distributive Interpretation (DI)}: Each and every student ate a pizza themselves.
				\end {xlist}
			\ex Dono gakusei-mo piza-o tabe-owat-ta.
			\begin {xlist} 
				\ex \textbf{Collective Interpretation (CI)}: *All the students ate a single pizza together.
				\ex \textbf{Distributive Interpretation (DI)}: Each and every student ate a pizza themselves.
			\end {xlist}
	\end {xlist}
\end {exe}
The sentences in (8) show the difference between using a Japanese quantifier (subete-no) that roughly translates to the English \emph{all}, and one (dono-N-mo) that roughly translates to the English \emph{every}. The reason behind writing a/the instead of a certain determiner for the object, pizza (piza-o), is because of the ambiguity that arises from both the collective and distributive interpretations. The \emph{the} aligns with the collective interpretation, which contains a wide scope to imply that \emph{All students ate the pizza} follows the definition in (9a-i). The \emph{a} aligns with the distributive interpretation, which contains a narrow scope to imply that \emph{All students ate a pizza} follows the definition in (9a-ii). Figure 5 shows what we have learned about for subete-no and dono-N-mo thus far.
\begin{figure}[h]
	\begin{center} \renewcommand*\arraystretch{1.2}
	\scalebox{1}[1]{\begin{tabular}[t]{|rrl||c|c|c|} \hline 
	\multicolumn{3}{|c||}{Japanese} & \sc{Distributive} & \sc{Collective}  \\[0.5ex]
  	 	\hline & Subete-no 		& & $\checkmark$ & $\checkmark$ \\
		\hline & Dono-N-mo		& & $\checkmark$ & $\dwork$ \\
   	 	\hline 
	\end{tabular}} \renewcommand*\arraystretch{1} \end{center}
	\vspace*{-5mm}
	\captionsetup{labelfont=bf}
	\caption[labelfont=bf]{Universal Quantifiers and Interpretations I (Japanese)}
\end{figure}

\subsection{Different Classes of Japanese Quantifiers}
% Introduce the crosslinguistic Wh+N+QP universal quantifier
% Talk about how this is DI and not CI
% Introduce two prenominal universal quantifiers (subete-no, zenbu-no)
% Talk about how this is DI and CI
% Draw out a simple table with DI / CI for dono-N-mo and subete-no, as well as zenbu-no
According to research done by (\cite{s1}) and (\cite{s1}), there are two types of universal quantifiers in Japanese. One type is the \emph{simple universal quantifier}, which is a quantifier that permits a variety of scope relations. Previously used terminology would imply that simple universal quantifiers are quantifiers that allow for both a collective and distributive interpretation. In Japanese, examples of this type of quantifier include \emph{subete-no}, \emph{zenbu-no}, and \emph{minna}, which all roughly translate to the English \emph{all}. 

Another type of universal quantifier in Japanese is a unique, crosslinguistic \emph{distributive-key universal quantifier}. The structure for this kind of universal quantifier follows a Wh+Noun+QP structure, which requires a noun to be stuck in between a Wh word (like what, when, or how) and a quantifier particle. This special class of quantifiers forces the subject NP within the Wh-N-QP structure to have a wide scope (\cite{s3}). This means that the collective interpretation is rendered ungrammatical, and only the distributive interpretation is accepted as a valid meaning for a sentence that uses a distributive-key universal quantifier.

\begin{figure}[h]
	\begin{center} \renewcommand*\arraystretch{1.2}
	\scalebox{1}[1]{\begin{tabular}[t]{|rrl||c|c|c|} \hline 
	\multicolumn{3}{|c||}{Japanese} & \sc{Distributive} & \sc{Collective}  \\[0.5ex]
  	 	\hline & Simple UQ 		& & $\checkmark$ & $\checkmark$ \\
		\hline & Distributive-key UQ & & $\checkmark$ & $\dwork$ \\
   	 	\hline 
	\end{tabular}} \renewcommand*\arraystretch{1} \end{center}
	\vspace*{-5mm}
	\captionsetup{labelfont=bf}
	\caption[labelfont=bf]{Universal Quantifiers and Interpretations II (Japanese)}
\end{figure}

\section{Testing Other Japanese Quantifiers for Ambiguity}
\subsection{Zenbu-no}
\emph{Zenbu-no} means all, and uses the same Kanji character for \emph{zen} as the \emph{sube} in the previously covered \emph{subete-no}. It is a distributive-key universal quantifier, and has the same qualities as \emph{subete-no}. 
\begin{exe}
	\ex 
	\begin{xlist}
		\ex[]{
		\gll Subete-no gakusei-wa piza-o tabe-owat-ta \\
			all-Gen student-Top pizza-Acc eat-Finish-Past. \\
			\trans `All students ate a/the pizza.' }\label{1b}
		\ex[]{
		\gll Zenbu-no gakusei-wa piza-o tabe-owat-ta \\
			all-Gen student-Top pizza-Acc eat-Finish-Past. \\
			\trans `All students ate a/the pizza.' }\label{1b}
	\end{xlist}
\end{exe}
The glossing in (10) shows the lack of any kind of difference between \emph{zenbu-no} and \emph{subete-no} from a semantic and syntactic standpoint. More analysis into the ontological reasons as to why these two different words exist reveals that \emph{zenbu-no} is a Japanese word with an onyomi reading, or a reading that is derived from Chinese pronunciations. On the other hand, \emph{subete-no} is a quantifier with a kunyomi reading, or a reading that follows the pronunciation for the original, native Japanese readings, hence the same Kanji character being used for \emph{zenbu} and \emph{subete}.

\subsection{Aru-N and Aru-N-tati}
\emph{Aru-N} is an existential quantifier that means \emph{some}. Unlike \emph{dareka}, which allows for both singular and plural regardless of the noun it may be referencing, \emph{aru-N} is a quantifier that can only be used with singular nouns when the noun is singular, and can only be used with plural nouns when the noun \emph{aru-N} is attached to has been pluralized and there is animacy. The pluralized form of \emph{aru-N} is \emph{aru-N-tati}, and only applies to animated nouns. To summarize, \emph{aru-N}'s meaning changes in terms of mass and count depending on the plurality and animacy of the noun that it is attached to.
\begin{exe}
	\ex 
	\begin{xlist}
		\ex[]{
		\gll Aru gakusei-ga zenbu-no kyoujyu-ni at-ta. \\
			some student-Nom every-Gen professor-Dat meet-Past. \\
			\trans `Some student met every professor.' }\label{1a}
		\ex[]{
		\gll Aru gakusei-tati-ga zenbu-no kyoujyu-ni at-ta. \\
			some student-Pl-Nom every-Gen professor-Dat meet-Past. \\
			\trans `Some students met every professor.' }\label{1b}
		\ex[]{
		\gll Zenbu-no kyoujyu-ni aru gakusei-ga at-ta. \\
			some student-Nom every-Gen professor-Dat meet-Past. \\
			\trans `Some student met every professor. (scrambled)' }\label{1a}
		\ex[]{
		\gll Zenbu-no kyoujyu-ni aru gakusei-tati-ga at-ta. \\
			some student-Pl-Nom every-Gen professor-Dat meet-Past. \\
			\trans `Some students met every professor. (scrambled)' }\label{1b}
		\ex[]{
		\gll Subete-no gakusei-ga aru-tatemono-wo nagame-ta. \\
			some student-Pl-Nom every-Gen professor-Dat meet-Past. \\
			\trans `All the students gazed at a building.' }\label{1b}
		\ex[*]{
		\gll Subete-no gakusei-ga aru-tatemono-tati-wo nagame-ta. \\
			some student-Pl-Nom every-Gen professor-Dat meet-Past. \\
			\trans `All the students gazed at the buildings.' }\label{1b}
	\end{xlist}
\end{exe}
As demonstrated in (11), \emph{Aru-N} seems to either have the English interpretation \emph{a}, \emph{the}, or \emph{some}, depending on the context in which it is used. However, using this with a pluralized noun requires the noun to be animated, otherwise the sentence becomes ungrammatical as seen in (11f). The scrambled versions of (11a) and (11b), listed as (11c) and (11d), are both ambiguous because \emph{zenbu-no} is set in the object position and moved towards the front. With the data above, we can infer that (11e) is not ambiguous, but a scrambled version of (11e) would be ambiguous.

\subsection{Sorezore-no}
\emph{Sorezore-no} is a universal quantifier that means \emph{each} or \emph{each of}. It is different from \emph{zenbu-no} and \emph{subete-no} because it only accepts a distributive interpretation. However, it does not follow the same structure as the other distributive-key universal quantifier, \emph{dono-N-mo}, as it does not follow the wh word + noun + quantifier particle rule set for \emph{dono-N-mo}. As discussed in section 2.1, a universal quantifier in the object position of a scrambled Japanese QP-QP sentence with an existential quantifier in the subject position should result in a sentence with two readings. Let us take a closer look at this phenomenom with \emph{sorezore-no}.
\begin{exe}
	\ex[]{
	\gll sorezore-no hantaa-wo aru kuma-ga  korosi-ta. \\
		each-Gen hunter-Acc some bear-Nom kill-Past. \\
		\trans `Some bear killed each hunter. (scrambled)' }\label{1a}
\end{exe}
\begin {exe}
	\ex Sorezore-no hantaa-wo aru kuma-ga korosi-ta. (scrambled)
		\begin {xlist} 
			\ex \textbf{Collective Interpretation (CI)}: Each hunter was killed by the same bear.
			\ex \textbf{Distributive Interpretation (DI)}: Each hunter was killed by a different bear.
		\end {xlist}
\end {exe}
\begin{equation}
	\text{CI:\:} \llbracket  \text{[Aru\:kuma-ga]}_{2}\:\text{[Sorezore-no\:hantaa-wo]}_{1}\:[e_{1}\:e_{2}\:\text{korosi-ta}] \rrbracket \quad
\end{equation}
\begin{equation}
	\text{DI:\:} \llbracket  \text{[Sorezore-no\:hantaa-wo]}_{1}\:\text{[aru\:kuma-ga]}_{2}\:[e_{1}\:e_{2}\:\text{korosi-ta}] \rrbracket \quad
\end{equation}
\begin{figure}[!h]
\centering
	\captionsetup{labelfont=bf}
	\caption[labelfont=bf]{Logical Form 1 (CI) and Logical Form 2 (DI) for (12)}
	\begin{tikzpicture}[scale = 0.7]
		\Tree [.t  [.<e,t>,t \edge[roof];\node(2){Aru\:kuma-ga_{2}}; ] [.t [.<e,t>,t \edge[roof];\node(4){sorezore-no\:hantaa-wo_{1}}; ] [.t \node(3){e_{1}}; [. <e,t> \node(1){e_{2}}; {korosi-ta} ] ] ] ]]
		\draw [thin, dashed, ->] (1) .. controls +(south:3) and +(south:4.5) .. (2);
		\draw [thin, dashed, ->] (3) .. controls +(south:2) and +(south:3.5) .. (4);
	\end{tikzpicture} \hspace{20pt}
	\begin{tikzpicture}[scale = 0.7]
		\Tree [.t  [.<e,t>,t \edge[roof];\node(2){Sorezore-no\:hantaa-wo_{1}}; ] [.t [.<e,t>,t \edge[roof];\node(4){Aru\:kuma-ga_{2}}; ] [.t \node(3){e_{1}}; [. <e,t> \node(1){e_{2}}; {korosi-ta} ] ] ] ]]		\draw [thin, dashed, ->] (1) .. controls +(south:3) and +(south:3.5) .. (4);
		\draw [thin, dashed, ->] (3) .. controls +(south:2) and +(south:3.5) .. (2);
	\end{tikzpicture}
\end{figure} \newline
Following the scrambled sentence in (12), the universal quantifier \emph{sorezore-no} placed in the front of the sentence allows for an ambiguous (both collective and distributive) reading given that an existential quantifier like \emph{aru-N} is used. When the sentence is not scrambled, the distributive interpretation applies, saying that the same bear killed each hunter. Note that the equations (5) and (6) refer to the placement of each quantifier phrase in the scrambled form of (12), \emph{e_{1} {e_2} killed}, or in English, \emph{e_{2} killed e{1}} (scrambled).

\subsection{Nan-*-ka-no-N}
\emph{Nan-*-ka-no-N} is an existential quantifier that means \emph{some *}. The * refers to a measure word that is used in combination with Wh, a QP (quantifier particle \emph{ka}), genitive case, and noun that the measure word refers to. In English, one would refer to words like \emph{water} or \emph{furniture}, which are invalid when counted without a measure word. On the other hand, in Japanese, numerals usually cannot quantify nouns by themselves. Although this has a similar structure to the distributive-key universal classifier \emph{dono-N-mo}, this is an existential quantifier; the rules that apply to \emph{nan-*-ka-no-N} are different. 
\begin {exe}
	\ex 
		\begin {xlist}
			\ex *There are three waters in the glass.
			\ex There are three \underline{drops of water} in the glass.
			\ex *There are two furniture in the room.
			\ex There are two \underline{pieces of furniture} in the room.
	\end {xlist}
\end {exe}
\begin{exe}
	\ex 
	\begin{xlist}
		\ex[]{
		\gll Aru gakusei-ga zenbu-no kyoujyu-ni at-ta. \\
			some student-Nom every-Gen professor-Dat meet-Past. \\
			\trans `Some student met every professor.' }\label{1a}
		\ex[]{
		\gll Aru gakusei-tati-ga zenbu-no kyoujyu-ni at-ta. \\
			some student-Pl-Nom every-Gen professor-Dat meet-Past. \\
			\trans `Some students met every professor.' }\label{1b}
	\end{xlist}
\end{exe}

\subsection{Summary}
The newly reviewed universal quantifiers and their traits can be seen in Figure 8. Interestingly enough, a quantifier like \emph{sorezore-no}, which does not follow the same crosslinguistically unique structure that the distributive-key quantifier \emph{dono-N-mo} has, despite being a distributive-key quantifier. 
\vs
\begin{figure}
\begin{center} \renewcommand*\arraystretch{1.2}
\scalebox{1}[1]{\begin{tabular}[t]{|rrl||c|c|c|} \hline 
\multicolumn{3}{|c||}{Japanese UQ} & \sc{Distributive} & \sc{Collective}  \\[0.5ex]
	\hline & Dono N-mo 	 & & $\checkmark$ & $\ast$ \\
	\hline & Sorezore-no 	 & & $\checkmark$ & $\ast$ \\
	\hline & Subete-no 		 & & $\checkmark$ & $\checkmark$ \\
	\hline & Zenbu-no 	& & $\checkmark$ & $\checkmark$ \\
	\hline 
\end{tabular}} \renewcommand*\arraystretch{1} 
\captionsetup{labelfont=bf}
\caption[labelfont=bf]{Universal Quantifiers and Interpretations III (Japanese)}
\end{center}
\end{figure}
\section{Conclusion}

\vs

\begin{thebibliography}{999}
	\bibitem[Lakoff(1972)]{s1}
	George Lakoff,
  	\emph{Linguistics and Natural Logic}.
  	Semantics of Natural Language, pp. 545-665,
  	1972.
  	\bibitem[Marsden(2009)]{s2}
	Heather Marsden,
  	\emph{Distributive Quantifier Scope in English-Japanese and Korean-Japanese Interlanguage}.
  	Language Acquisition, vol. 16, no. 3, pp. 135-177,
  	2009.
  	\bibitem[Bach(1995)]{s3}
	Emmon Bach, Eloise Jelinek, Angelika Kratzer, Barbara H. Partee,
  	\emph{Quantification in Natural Languages}.
  	Kluwer Academic Publishers, pp. 340-342,
  	1995.
  	\bibitem[Kuno(1999)]{s4}
	Susumu Kuno, Kenichi Takami, Yuru Wu,
  	\emph{Quantifier Scope in English, Chinese, and Japanese}.
  	Linguistic Society of America, vol. 75, no. 1, pp. 105-111,
  	1999.
\end{thebibliography}
\bibliographystyle{chicago}
\end{document} 